%===============================================================================
%*** PYTANIA I ODPOWIEDZI ******************************************************
%===============================================================================

\section{\textcolor{blue}{RS-232}}
\subsection*{Prawda/Fałsz}
\begin{itemize}
	\item \textcolor{nie}{RS-232 jest portem przeznaczonym do synchronicznej transmisji znakowej. Generator taktu odpowiedzialny za wyprowadzanie znaków typowo ustawiany jest na: 1200bd, 2400bd, 4800bd, 9600bd, 19200bd.} \\
	{\small \emph{RS-232 jest portem przeznaczonym do asynchronicznej transmisji znakowej. Da się sztucznie stworzyć synchroniczną transmisję.}}
	
	\item \textcolor{tak}{Linie kontrolne w interfejsie RS-232 to: DTR, DSR, RTS, CTS, RI, DCD. Pary DTR/DSR i RTS/CTS wykorzystywane są do realizacji handshake'u w połączeniach bezmodemowych.} \\ {\small \emph{Tak, te pary linii mogą być wykorzystywane do handshake podczas gdy RxD i TxD zajmują się przesyłem danych.}}
	
	\item \textcolor{nie}{Transakcja w systemie MODBUS składa się z zapytania (query) wysyłanego przez stację Slave i odpowiedzi odsyłanej przez stację Master.} \\
	{\small \emph {Jest odwrotnie - zapytanie wysyła Master, a odpowiedź odsyła Slave.}}
	
	\item \textcolor{nie}{W trybie transmisji ASCII znacznikiem początku ramki jest znak ':', a kooca ramki para znaków CR LF. W trybie transmisji RTU znacznikiem początku ramki jest znak 'Ctrl-A', a kooca para znaków CTRL-Y CTRL-Z.} \\
	{\small \emph{Zdanie jest poprawne dla ASCII. Dla RTU, znacznikiem początku i końca ramki jest przerwa o długości minimum 4T, gdzie T jest czasem trwania jednego znaku.}}
	
	\item \textcolor{nie}{Standard RS-232 transmituje znaki synchronicznie, bity w znakach [asynchronicznie]} \\
	{\small \emph{Ostatnie słowo ucięte, więc spekuluję że tak właśnie było napisane. To nieprawda, jest odwrotnie.}}
	
	\item \textcolor{tak}{Standard RS-422 pozwala na osiągnięcie szybkości 10MBodów na odległości 100m.} \\
	{\small \emph{IMO pozwala, na slajdzie 12 jest napisane że 10 Mbd przy zasięgu DO 100m - czyli 100m chyba też.}}
	
	\item \textcolor{tak}{Liniami kontrolnymi w RS-232 nie są linie TxD, RxD, SG.} \\
	{\small \emph{Owszem, TxD i RxD są liniami danych, a SG to po prostu masa.}}
	
	\item \textcolor{tak}{System MODBUS składa się z faz zapytania i odpowiedzi.} \\
	{\small \emph{Tak właśnie jest.}}
	
	\item {W systemie MODBUS}
	\begin{itemize}
		\item \textcolor{tak}{Obowiązuje master/slave.} \\
		{\small \emph{Pewnie, a w dodatku Slave'ów może być wielu.}}
		
		\item \textcolor{tak}{Prędkości transmisji wynoszą od 1200 do 19200bd.} \\
		{\small \emph{Jak najbardziej.}}
		
		\item \textcolor{tak}{Ramka w ASCII może mieć format 7N2 (lub np. 7E1, 7O1).} \\
		{\small \emph{Tak, patrz warstwa fizyczna MODBUS.}}
		
		\item \textcolor{tak}{Ramka w RTU może mieć format 8N2 *(lub np. 8E1, 8O1).} \\
		{\small \emph{Tak, patrz warstwa fizyczna MODBUS.}}
	\end{itemize} 
	
	
	\item \textcolor{tak}{W trybie transmisji RTU jest kontrola błędów CRC.} \\
	{\small \emph{Tak, jest elementem budowy ramki RTU.}}
	
	\item \textcolor{nie}{Bit kontrolny w RS-232 zależy od bitu danych i bitu stopu.} \\
	{\small \emph{Bit kontrolny słuzy do kontroli parzystości/nieparzystości, nie ma związku z bitem stopu.}}
	
	\item \textcolor{nie}{Za pomocą RS-232 możemy połączyć ze sobą 2 stacje DCE} \\
	{\small \emph{Połączyć możemy dwie stacje DTE, lub DTE z DCE. Dwie stacje DCE łączą się za pomocą łącza telefonicznego.}}
	
	\item \textcolor{tak}{W MODBUS kontrola błędów jest realizowana za pomocą LRC lub CRC.} \\
	{\small \emph{Tak, LRC wykorzystywane jest w trybie ASCII, CRC w trybie RTU.}}
	
	\item \textcolor{nie}{Do portu RS 485 można podłączyć tylko jedno urządzenie, ale za to obsługiwać go z dużo większą szybkością i na większą odległość niż jest to możliwe w przypadku interfejsu RS 232.} \\
	{\small \emph{Można podłączyć do 32 stacji.}}
	
	\item \textcolor{nie}{Format ramki w protokole Modbus jest następujący: znacznik początku ramki, adres urządzenia slave, adres mastera, pole danych, znacznik końca ramki.} \\
	{\small \emph{Opis nie pasuje ani do trybu ASCII, ani RTU}}
	
	\item \textcolor{nie}{RS 232 jest portem przeznaczonym dla asynchronicznej transmisji znakowej, realizowanej zazwyczaj w trybie dupleksowym, czyli dwukierunkowej transmisji niejednoczenej (naprzemiennej)} \\
	{\small \emph{Tryb dupleksowy jest równoczesny, to półdupleksowy jest niejednoczesny.}}
	
	\item \textcolor{tak}{W interfejsie RS 232 linie TxD i RxD służą do transmisji znaków, natomiast DTR, RTS to wyjścia kontrolne, a DSR, CTS, RI i DCD to wejścia kontrolne.} \\
	{\small \emph{Indeed}}
	
	\item \textcolor{tak}{Multipleksowanie urządzeń ze znakowym portem asynchronicznym pozwala na ich kontrolę poprzez jeden port RS-232.} \\
	{\small \emph{Żeby kontrolować kilka urządzeń z jednego portu potrzebny jest koncentrator. Jeśli "używanie koncentratora" równa się "multipleksowanie", to PRAWDA.}}
	
	\item \textcolor{tak}{Węzeł podrzędny w systemie MODBUS po wykryciu błędu w komunikacie wysyła potwierdzenie negatywne	do węzła nadrzędnego.} \\
	{\small \emph{W odpowiedzi pole to jest wykorzystywane do pozytywnego lub negatywnego potwierdzenia wykonania polecenia.}}
	
	\item \textcolor{tak}{Czy w trybie ASCII systemu MODBUS każdy bajt wysyłany jest jako znak z przedziału 0x00, 0xFF?} \\
	{\small \emph{Bajt dzielimy na 2 części i wysyłamy jako 2 znaki z przedziału 0-9 i Ah-Fh}}
		
	
\end{itemize}


\section{\textcolor{blue}{USB}}
\subsection*{Prawda/Fałsz}
\begin{itemize}
	
	\item \textcolor{tak}{Kontrola urządzenia USB odbywa się poprzez zapisy komunikatów do bufora o numerze 0 i odczycie informacji statusowych z bufora o numerze 0.} \\
	{\small \emph{Zgadza się.}}
	
	\item \textcolor{nie}{W przypadku błędu transmisji każda transakcja USB jest powtarzana, ponieważ niedopuszczalne jest przekazywanie danych przekłamanych.} \\
	{\small \emph{Transakcje izochroniczne nie są powtarzane w przypadku błędu transmisji.}}
	
	\item \textcolor{tak}{Hub nie dopuszcza ruchu full speed do portów, do których są podłączone urządzenia low speed.} \\
	{\small \emph{Tak, urządzenie lowspeed blokuje możliwość włączenia fullspeed na całym porcie.}}
	
	\item \textcolor{nie}{Reset portu USB polega na rekonfiguracji hosta, po której host zapisuje tablicę deskryptorów do urządzenia podłączonego do tego portu.} \\
	{\small \emph{Reset portu USB polega na rekonfiguracji urządzenia. W następującej procedurze enumeracji między innymi dochodzi do odczytu tablicy deskryptorów z urządzenia przez host.}}
	
	\item \textcolor{nie}{Typowa transakcja USB składa się z pakietów żądania i odpowiedzi, z których każdy potwierdzany jest osobnym potwierdzeniem.} \\
	{\small \emph{Typowa transakcja USB składa sie z pakietów token, data i handshake. Transakcje izochroniczne nie są potwierdzane.}}
	
	\item \textcolor{nie}{W systemie USB urządzenia zgłaszają żądania do hosta, który je kolejkuje i następnie obsługuje w kolejności pojawiania się zgłoszenia.} \\
	{\small \emph{Urządzenia nie zgłaszają żądania, tylko są odpytywane przez hosta. Host nie tworzy jednej kolejki, tylko w miarę możliwości stara się obsługiwać wszystkie urządzenia jednocześnie, równomiernie, zapobiegając zawłaszczeniu.}}
	
	\item \textcolor{nie}{W USB można połączyd kaskadowo do 5 hubów, korzystających z zasilania magistralowego} \\
	{\small \emph{Podłączyć je można tylko korzystając z zasilania zewnętrznego lub hybrydowego. Przy zasilaniu magistralowym zabraknie zasilania już na drugim hubie. Co więcej, należy mieć na uwadze maksymalne dopuszczalne opóźnienie sygnału, które przy przejściu przez 5 hubów jest osiągane - 350ns. Urządzenia podpięte do 5'tego huba mogą nie działać poprawnie.}}
	
	\item \textcolor{nie}{Mechanizm data toggle w USB służy do przywracania synchronizacji pomiędzy hostem i urządzeniem, utraconej na skutek wystąpienia błędów w pakietach danych.} \\
	{\small \emph{Mechanizm data toggle zabezpiecza przed utratą synchronizacji pomiędzy hostem i urządzeniem na skutek błędu w potwierdzeniu odsyłanym przez odbiorcę.}}
	
	\item \textcolor{tak}{Host kontroler USB komunikuje się z interfejsem magistrali USB urządzenia peryferyjnego za pomocą fizycznego kanału komunikacyjnego.} \\
	{\small \emph{Tak, używamy kabelka.}}
	
	\item \textcolor{nie}{Kamera internetowa może przesyłać obraz do komputera za pomocą transferu izochronicznego z szybkością LowSpeed w interfejsie USB.} \\
	{\small \emph{Z tabelki można wyczytać, że dla transferu izochronicznego nie można wykorzystać szybkości LowSpeed.}}
	
	\item \textcolor{tak}{Pakiety USB przesyłane z szybkością LowSpeed muszą byd poprzedzone pakietem preambuły} \\
	{\small \emph{Tak, jest on charakterystyczny dla pakietów przesyłanych z szybkością LowSpeed}}
	
	\item \textcolor{nie}{Urządzenie peryferyjne USB 2.0 może być podłączone do host kontrolera za pośrednictwem maksymalnie sześciu hubów.} \\
	{\small \emph{Aby spełnić normę (ograniczenie czasowe oczekiwania na odpowiedź), można podłączyd za pośrednictwem maksymalnie 5 hubów.}}
	
	\item \textcolor{nie}{Pole PID w pakiecie USB zabezpieczone jest 16-bitową sumą kontrolną CRC.} \\
	{\small \emph{Pole PID zabezpieczone jest 4-bitowym polem kontroli, będącym prostą negacją bitów pola PID.}}
	
	\item \textcolor{nie}{Do portu dolnego huba podłączane mogą byd tylko wtyki USB typu B.} \\
	{\small \emph{Tylko wtyki typu A.}}
	
	\item \textcolor{nie}{Transakcja dzielona w USB 1.1 składa sie z dwóch części: SSPLIT i CSPLIT.} \\
	{\small \emph{Takie czary dopiero w USB 2.0}}
	
	\item \textcolor{nie}{W przypadku połączenia USB HighSpeed wykonywane jest podparcie linii D- do Vcc za pośrednictwem rezystora 1,5k.} \\
	{\small \emph{Po podłączeniu urządzenia High Speed wpierw jest ono identyfikowane jako Full Speed, więc wykonywane jest podparcie linii D+ do Vcc za pośrednictwem rezystora 1,5k. Następnie, poprzez chirp ("dwierkanie") host i urządzenie ustalają, czy możliwa jest komunikacja w trybie High Speed. Jeśli tak, usuwane jest podparcie przez rezystor, a obwód zamykany jest terminatorami.}}
	
	\item \textcolor{nie}{W kodowaniu NRZI co sześć jedynek jest wstawiany bit synchronizacji "0".} \\
	{\small \emph{Pomieszane pojęcia. W kodowaniu NRZI nie występuje dodawanie bitu synchronizacji. Proces ten nazywa się bit stuffing. Zdanie było by poprawne, gdyby brzmiało np. W kodowaniu NRZI z bit stuffingiem co sześć.}}
	
	\item \textcolor{nie}{Transakcje kontrolna i przerwaniowa w USB 1.1 są transakcjami aperiodycznymi z gwarantowanym pasmem w ramach jednej mikroramki.} \\
	{\small \emph{Transakcja kontrola jest transakcją aperiodyczną. Transakcja przerwaniowa jest transakcją periodyczną.}}
	
	\item \textcolor{tak}{W kontrolerze OHC transakcje izochroniczne są porządkowane/kolejkowane w drzewo/strukturę drzewiastą.} \\
	{\small \emph{Tak, OHC wykorzystuje strukturę drzewa, a UHC tablicę wskaźników (listę podwieszaną).}}
	
	\item \textcolor{tak}{Standard USB 2.0 wymaga skręconych, ekranowanych kabli.} \\
	{\small \emph{Well, High speed all the way, więc wymaga}}
	
	\item \textcolor{nie}{Transfer kontrolny i przerwaniowy są transferami aperiodycznymi.} \\
	{\small \emph{Było podobne pytanie. Transfer kontrolny jest aperiodyczny, transfer przerwaniowy jest periodyczny.}}
	
	\item \textcolor{tak}{Wielowarstwowa architektura USB 2.0 składa się z 3 warstw.} \\
	{\small \emph{Tak - warstwa interfejsu magistrali USB, warstwa urządzenia USB, warstwa funkcji urządzenia}}
	
	\item \textcolor{tak}{W porcie USB dane są dzielone na transakcje.} \\
	{\small \emph{Dane w ramce są dzielone na transakcje, więc tak}}
	
	\item \textcolor{nie}{Hub podłączony do portu USB ma obciążalność 100uA.} \\
	{\small \emph{Hub podłączony do portu USB bez własnego zasilania (zasilanie magistralowe) ma obciążalnośd dla portów dolnych do 100mA na port (maksymalną 400mA na cały hub). Hub z zasilaniem zewnętrznym lub hybrydowym ma obciążalnośd do 500mA na port.}}
	
	\item {W systemie USB do mechanizmów kontroli danych należą:}
	\begin{itemize}
		\item \textcolor{tak}{Przełączanie pakietów danych} \\
		{\small \emph{Tzw. Data Toggle}}
		
		\item \textcolor{tak}{Wykrywanie braku aktywności na linii danych;}
		
		\item \textcolor{nie}{Zabezpieczenie znacznika SOF lub EOF} \\
		{\small \emph{Reakcją jest natomiast objęte wystąpienie fałszywego znacznika kooca pakietu (false EOP)}}
		
		\item \textcolor{nie}{kodowanie LRC} \\
		{\small \emph{Pakiety zabezpieczone są kodowaniem CRC.}}
	\end{itemize}
	
	\item \textcolor{nie}{Wydajnośd dolnego portu (USB 2.0) wynosi 500mA.} \\
	{\small \emph{Nie wiadomo. Zasilany Hub może wystawić te 500mA, ale niezasilany już tylko 100mA}}
	
	\item \textcolor{tak}{USB 2.0 ma parę przewodów ekranowanych.} \\
	{\small \emph{Taki upgrade.}}
	
	\item \textcolor{nie}{W kodowaniu NZR wstawia się dodatkowe bity synchroniczne.} \\
	{\small \emph{Dodatkowe bity synchroniczne wstawia się w kodowaniu NRZI}}
		
	\item \textcolor{nie}{Urządzenie USB 2.0 może zasygnalizować swoją niegotowość do zapisu danych z szybkością High-Speed wysyłając pakiet PING-NYET.} \\
	{\small \emph{Wychodzi na to, że niegotowość zgłasza samym NYET? Pyta – PING, odpowiada (niegotowość) NYET. I Tak cały czas, chyba że dostanie ACK. ACK – wykonanie transakcji OUT. NYRT – host kontynuuje wysyłanie zapytań PING}}
	
	\item \textcolor{nie}{W systemie deskryptorów urządzenia USB może wystąpić kilka deskryptorów urządzenia, konfiguracji, interfejsów I punktów końcowych.} \\
	{\small \emph{Deskryptor urządzenia może być jeden. Innych – konfiguracji, interfejsu, końcowych może być więcej.}}
	
	\item \textcolor{tak}{Hub USB ma przerwaniowy punkt końcowy, który wykorzystuje do powiadamiania hosta o podłączeniu	urządzenia USB do któregoś z jego portów dolnych.} \\
	{\small \emph{Chyba.}}
	
	\item \textcolor{nie}{Na wierzchołku wielopoziomowego, hierarchicznego układu deskryptorów USB znajduje się deskryptor konfiguracji.}
	{\small \emph{Na szczycie znajduje się pojedynczy deskryptor urządzenia.}}
	
	\item \textcolor{nie}{Transfer masowy I izochroniczny USB 1.1 są przykładami transferów aperiodycznych z zagwarantowanym pasmem w ramach jednej mikroramki.} \\
	{\small \emph{Izochroniczny jest periodyczny, masowy nie ma zagwarantowanego pasma (wg tabelki z	prędkościami)}}
	
	\item \textcolor{nie}{W deskryptorze konfiguracji USB jest jakiś pole statusowe, które mówi o maksymalnym poborze prądu. Dla wartości 50 urządzenie pobiera 50mA.} \\
	{\small \emph{Pole to jest tak skonstruowane, żeby wartość zmieściła się w jednym bajcie, ze skokiem co 2mA. Dlatego urządzenie, które zgłasza, że 50 może zasysać maksymalnie 100mA.}}
	
	\item \textcolor{nie}{Uszeregowanie transakcji w USB nie zależy od implementacji kontrolera.} \\
	{\small \emph{W OHC przerwaniowe są w strukturze drzewa, a w UCH listy podwieszanej, co ma wpływ na uszeregowanie (do sprawdzenia).}}
	
	\item \textcolor{nie}{Host może zasygnalizować chęć zapisu danych do urządzenia wysyłając pakiet NYET do urządzenia USB 2.0, które z kolei odpowiada pakietem PING jeśli jest gotowe do zapisu.} \\
	{\small \emph{To host posyła PING - zapytanie, czy urządzenie jest gotowe do zapisu. Te odsyła ACK - gotowe, lub NYET - jeszcze nie.}}
	
\end{itemize}

\section{\textcolor{blue}{IEEE 1394 Firewire}}
\subsection*{Prawda/Fałsz}
\begin{itemize}
	
	\item \textcolor{tak}{Po resecie w systemie FireWire (IEEE1394) wykonywane są procedury TREEID  i SELFID .} \\
	{\small \emph{Po resecie następuje „TREEID” odpowiedzialne za ustalenie węzła głównego a później „SELFID” odpowiedzialne za rozesłanie adresów do poszczególnych portów.}}
	
	\item \textcolor{tak}{Transakcja dzielona IEEE1394 umożliwia wykorzystanie magistrali przez inne węzły po zakooczeniu subakcji żądania (request), a przed wysłaniem odpowiedzi (response).} \\
	{\small \emph{Tak, pomiędzy żądaniem a odpowiedzą magistrala jest wolna i można ją wykorzystać}}
	
	\item \textcolor{nie}{Transakcje asynchroniczne IEEE1394 są uprzywilejowane w stosunku do transakcji izochronicznych i w związku z tym zawsze wykonywane są na początku cyklu.} \\
	{\small \emph{Jest odwrotnie. To izochroniczne są uprzywilejowane nad asynchronicznymi.}}
	
	\item \textcolor{tak}{Urządzenie IEEE 1394 będące konsumentem zasilania może posiadad co najwyżej jedno 6-kontaktowe gniazdo IEEE 1394.} \\
	{\small \emph{Tak po prostu jest.}}
	
	\item \textcolor{tak}{Numery węzłów IEEE 1394 nadawane są podczas procedury samoidentyfikacji na podstawie wartości wewnętrznych liczników odebranych pakietów SelfID.} \\
	{\small \emph{Tak, SELFID przypisuje każdemu węzłowi unikatowy identyfikator pełniący rolę adresu, a następnie rozsyła je w formie pakietów selfid z każdego portu do wszystkich pozostałych węzłów.}}
	
	\item \textcolor{tak}{Pakiet potwierdzenia odbioru asynchronicznego pakietu żądania zapisu bloku danych w pierwszej fazie transakcji asynchronicznej IEEE 1394 zawiera sumę kontrolną w formie parzystości.} \\
	{\small \emph{Tak, zawiera kod potwierdzenia i parzystość - patrz budowa pakietów}}
	
	\item \textcolor{nie}{W IEEE 1394 pakiet nowego cyklu (SCP) jest wysyłany ZAWSZE co 125us.} \\
	{\small \emph{Izochroniczne owszem, ale asynchroniczne w ramach interwału równych szans, którego długość zależy od liczby węzłów asynchronicznych jednocześnie żądających dostępu do łącza.}}
	
	\item \textcolor{tak}{IEEE 1394 posiada osobne pary ekranowanych przewodów (dla) TPA i TPB.} \\
	{\small \emph{Tak wynika z przekroju budowy}}
	
	\item \textcolor{tak}{Pole adresowe w IEEE1394 składa się z numeru magistrali (10b), numeru węzła (6b) i adresu w węźle (48b).} \\
	{\small \emph{Wszystko się zgadza.}}
	
	\item \textcolor{nie}{W systemie IEEE1394 węzeł A o szybkości S100 połączono z węzłem B o szybkości S200 za pośrednictwem węzła C o szybkości S400, co zwiększyło szybkość transmisji pomiędzy węzłami A i B w stosunku do ich połączenia bezpośredniego.} \\
	{\small \emph{Nie, węzeł A wciąż nadaje z szybkością S100, a teraz dodatkowo musi przejść przez węzeł C}}
	
	\item \textcolor{nie}{W interfejsie IEEE1394 przerwa pomiędzy subakcjami transakcji asynchronicznych jest mniejsza od przerwy pomiędzy transakcjami izochronicznymi, co powoduje ich uprzywilejowanie podczas arbitrażu.} \\
	{\small \emph{Izochroniczne mają pierwszeństwo, to raz. Dodatkowo przerwa pomiędzy izochronicznymi zazwyczaj jest krótsza od tej pomiędzy asynchronicznymi.}}
	
	\item \textcolor{nie}{Biorący udział w arbitrażu asynchronicznym węzeł A (dotyczy interfejsu IEEE 1394) nie może uzyskać dostępu do łącza, bo zdominował je asynchroniczny węzeł B, który ciągle wygrywa arbitraż.} \\
	{\small \emph{Kolejność przydziału zależy od położenia węzła w drzewie systemu. Węzły położone bliżej korzenia uzyskają dostęp przed węzłami bardziej oddalonymi. Łączny czas wykonania subakcji przez wszystkie węzły nazywa się fairness interval. W tym czasie każdy węzeł uzyska dostęp do magistrali, aby wykonać subakcję. Dostęp do magistrali węzeł zarządzający przekazuje “rotacyjnie”. Następna subakcja będzie mogła być wykonana w kolejnym interwale równych szans.}} \\
	
	\item \textcolor{nie}{Odbiorca transakcji asynchronicznej wygrał arbitraż I odesłał pakiet odpowiedzi. Inicjator transakcji odebrał odpowiedź I oczekuje na wygranie arbitrażu w celu odesłania pakietu potwierdzenia.} \\
	{\small \emph{Nie, istnieje coś takiego jak dostęp natychmiastowy - potwierdzenie wysyła się za pośrednictwem warstwy PHY, która nie rywalizuje o dostęp do magistrali.}}
	
	\item \textcolor{tak}{W przypadku transakcji dołączanych (interfejs IEEE 1394) nie trzeba rywalizować o dostęp do magistrali w celu odesłania odpowiedzi.} \\
	{\small \emph{Transakcja dołączana dostarcza dane przy potwierdzeniu, a ono nie wymaga arbitrażu.}}
	
	\item \textcolor{nie}{W interfejsie IEEE 1394 szansa wygrania arbitrażu wzrasta wraz ze wzrostem odległości (mierzone liczbą węzłów pośredniczących) węzła ubiegającego się o dostęp do magistrali od korzenia.} \\
	{\small \emph{Jest na odwrót – bliżej korzenia, większe szanse.}}
	
	\item \textcolor{nie}{Wartość przerw i ograniczeń czasowych w interfejsie IEE 1394 są “na sztywno” określone przez standard i nie mogą być korygowane.} \\
	{\small \emph{Przerwa pomiędzy subakcjami transakcji asynchronicznej może być regulowana.}}
	
	\item \textcolor{nie}{W przykładowym interfejsie IEEE 1394 występują tylko transakcje asynchroniczne. Jeżeli uczestnikiem (inicjatorem lub odbiorcą) transakcji asynchronicznej jest korzeń, to zawsze wygra on arbitraż jako pierwszy.} \\
	{\small \emph{Zgodnie z zasadą "Bliżej korzenia, większe szanse" korzeń wygrywa wszystko. Jednak występują transakcje izochroniczne i asynchroniczne.}}
	
	\item \textcolor{tak}{Kontroler cyklu musi być korzeniem w topologii IEE1394, bo musi wysyłać sygnał CSP.} \\
	{\small \emph{Chyba.}}
	
	\item \textcolor{tak}{Węzeł w IEE1394, który zainicjalizował transakcje asynchroniczną może być odbiorcą transakcji zainicjalizowanej przez inny węzeł w tym samym Interwale Równych Szans.} \\
	{\small \emph{Raczej tak, węzły asynchroniczne do wykonania transakcji nie wymagają alokacji pasma (subakcja). Also - zarządzanie magistralą stara się umożliwić jednoczesne jej wykorzystanie przez różne transfery.}}
	
\end{itemize}


\section{\textcolor{blue}{IEEE-488 i SCPI}}
\subsection*{Prawda/Fałsz}
\begin{itemize}
	
	\item \textcolor{tak}{GPIB (IEEE-488) jest interfejsem równoległym, opartym na 8-bitowej, 2 kierunkowej magistrali danych i 8 sygnałach sterujących: REN, IFC, ATN, SRQ, EOI, NRFD, NDAC, DAV} \\
	{\small \emph{Tak, bity wysyła się ósemkami, stąd m. in. podaje się prędkość w bajtach na sekundę}}
	
	\item \textcolor{nie}{SCPI to język programowania na bazie języka C wyposażony w biblioteki funkcji sterujących urządzeniami pomiarowo-kontrolnymi.} \\
	{\small \emph{SCPI jest językiem kontroli urządzeo (i nie bazuje na C).}}
	
	\item \textcolor{tak}{Znak ':' w rozkazach SCPI reprezentuje przejście pomiędzy poziomami w rozgałęzionej strukturze subsystemu, natomiast prefiks '*' oznacza rozkaz wspólny.} \\
	{\small \emph{Tak, dwukropek służy do precyzowania zapytania, gwiazdka jako nagłówek komunikatu wspólnego.}}
	
	\item \textcolor{nie}{System statusowy urządzenia SCPI składa się tylko z jednego, 8-bitowego rejestru, w którym bit B6 jest zgłoszeniem żądania obsługi.} \\
	{\small \emph{Składa się z minimum dwóch rejestrów, których układ jest wielopoziomowy (hierarchiczny).}}
	
	\item \textcolor{nie}{Kontrola szeregowa I kontrola równoległa to mechanizmy automatycznego wykrywania urządzeń podłączonych do systemu IEEE 488.} \\
	{\small \emph{Kontrola szeregowa I równoległa służą do identyfikacji urządzeń zgłaszających żadanie obsługi.}}
	
	\item \textcolor{nie}{Maska związana z bajtem statusowym SCPI służy do blokowania ustawiania wybranych bitów bajtu statusowego.} \\
	{\small \emph{maską związaną z bitem statusowym jest rejestr maski żądania obsługi, który odpowiada za selekcję bitów powodujących zgłoszenie żądania, ale nie ma on wpływu na stan samych bitów w bajcie statusowym. Bajt statusowy jest rejestrem zbiorczym swoich rejestrów nadrzędnych, więc jego wartość zależy do wartości tamtych rejestrów i ich masek.}}
	
\end{itemize}

\section{\textcolor{blue}{Inne}}
\subsection*{Prawda/Fałsz}
\begin{itemize}
	\item \textcolor{nie}{Interfejsy USB, IEEE1394 oraz RS-232 udostępniają zasilanie systemowe i mają mechanizmy zarządzania zasilaniem.} \\
	{\small \emph{USB i IEEE-1394 owszem, ale nie RS-232.}}
	
\end{itemize}

\newpage