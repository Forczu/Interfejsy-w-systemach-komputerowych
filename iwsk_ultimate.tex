% !TeX spellcheck = pl_PL
\documentclass[a4paper,twoside]{article}
\usepackage{polski}
\usepackage[utf8]{inputenc}
\usepackage{graphicx}
\usepackage{amsmath}

\usepackage[unicode, bookmarks=true]{hyperref} %do zakładek
\usepackage{tabto} % do tabulacji
\NumTabs{6} % globalne ustawienie wielkosci tabulacji
\usepackage{array}
\usepackage{multirow}
\usepackage{array}
\usepackage{dcolumn}
\usepackage{bigstrut}
\usepackage{color}
\usepackage[usenames,dvipsnames]{xcolor}
\usepackage{pdfpages}


\setlength{\textheight}{24cm}
\setlength{\textwidth}{15.92cm}
\setlength{\footskip}{10mm}
\setlength{\oddsidemargin}{0mm}
\setlength{\evensidemargin}{0mm}
\setlength{\topmargin}{0mm}
\setlength{\headsep}{5mm}

\newcolumntype{M}[1]{>{\centering\arraybackslash}m{#1}}
\newcolumntype{N}{@{}m{0pt}@{}}

\graphicspath{ {./images/} }

\definecolor{nie}{RGB}{178,34,34}
\definecolor{tak}{RGB}{0,120,0}

% === Reset inkrementacji sekcji przy nowym parcie === %
\usepackage{titlesec}

\makeatletter
\@addtoreset{section}{part}
\makeatother
\titleformat{\part}[display]
{\normalfont\LARGE\bfseries\centering}{}{0pt}{}


\begin{document}
\bibliographystyle{plain}



\begin{titlepage}
\title{\huge Interfejsy w Systemach Komputerowych - ULTIMATE}
\author{\large SonMati \\ Ervelan \\ Doxus}
\maketitle
\end{titlepage}

%===============================================================================
%*** PYTANIA I ODPOWIEDZI ******************************************************
%===============================================================================
\part{Pytania i odpowiedzi}

\section{\textcolor{blue}{RS-232}}
\subsection*{Prawda/Fałsz}
\begin{itemize}
	\item \textcolor{nie}{RS-232 jest portem przeznaczonym do synchronicznej transmisji znakowej. Generator taktu odpowiedzialny za wyprowadzanie znaków typowo ustawiany jest na: 1200bd, 2400bd, 4800bd, 9600bd, 19200bd.} \\
	{\small \emph{RS-232 jest portem przeznaczonym do asynchronicznej transmisji znakowej. Da się sztucznie stworzyć synchroniczną transmisję.}}
	
	\item \textcolor{tak}{Linie kontrolne w interfejsie RS-232 to: DTR, DSR, RTS, CTS, RI, DCD. Pary DTR/DSR i RTS/CTS wykorzystywane są do realizacji handshake'u w połączeniach bezmodemowych.} \\ {\small \emph{Tak, te pary linii mogą być wykorzystywane do handshake podczas gdy RxD i TxD zajmują się przesyłem danych.}}
	
	\item \textcolor{nie}{Transakcja w systemie MODBUS składa się z zapytania (query) wysyłanego przez stację Slave i odpowiedzi odsyłanej przez stację Master.} \\
	{\small \emph {Jest odwrotnie - zapytanie wysyła Master, a odpowiedź odsyła Slave.}}
	
	\item \textcolor{nie}{W trybie transmisji ASCII znacznikiem początku ramki jest znak ':', a kooca ramki para znaków CR LF. W trybie transmisji RTU znacznikiem początku ramki jest znak 'Ctrl-A', a kooca para znaków CTRL-Y CTRL-Z.} \\
	{\small \emph{Zdanie jest poprawne dla ASCII. Dla RTU, znacznikiem początku i końca ramki jest przerwa o długości minimum 4T, gdzie T jest czasem trwania jednego znaku.}}
	
	\item \textcolor{nie}{Standard RS-232 transmituje znaki synchronicznie, bity w znakach [asynchronicznie]} \\
	{\small \emph{Ostatnie słowo ucięte, więc spekuluję że tak właśnie było napisane. To nieprawda, jest odwrotnie.}}
	
	\item \textcolor{tak}{Standard RS-422 pozwala na osiągnięcie szybkości 10MBodów na odległości 100m.} \\
	{\small \emph{IMO pozwala, na slajdzie 12 jest napisane że 10 Mbd przy zasięgu DO 100m - czyli 100m chyba też.}}
	
	\item \textcolor{tak}{Liniami kontrolnymi w RS-232 nie są linie TxD, RxD, SG.} \\
	{\small \emph{Owszem, TxD i RxD są liniami danych, a SG to po prostu masa.}}
	
	\item \textcolor{tak}{System MODBUS składa się z faz zapytania i odpowiedzi.} \\
	{\small \emph{Tak właśnie jest.}}
	
	\item {W systemie MODBUS}
	\begin{itemize}
		\item \textcolor{tak}{Obowiązuje master/slave.} \\
		{\small \emph{Pewnie, a w dodatku Slave'ów może być wielu.}}
		
		\item \textcolor{tak}{Prędkości transmisji wynoszą od 1200 do 19200bd.} \\
		{\small \emph{Jak najbardziej.}}
		
		\item \textcolor{tak}{Ramka w ASCII może mieć format 7N2 (lub np. 7E1, 7O1).} \\
		{\small \emph{Tak, patrz warstwa fizyczna MODBUS.}}
		
		\item \textcolor{tak}{Ramka w RTU może mieć format 8N2 *(lub np. 8E1, 8O1).} \\
		{\small \emph{Tak, patrz warstwa fizyczna MODBUS.}}
	\end{itemize} 
	
	
	\item \textcolor{tak}{W trybie transmisji RTU jest kontrola błędów CRC.} \\
	{\small \emph{Tak, jest elementem budowy ramki RTU.}}
	
	\item \textcolor{nie}{Bit kontrolny w RS-232 zależy od bitu danych i bitu stopu.} \\
	{\small \emph{Bit kontrolny słuzy do kontroli parzystości/nieparzystości, nie ma związku z bitem stopu.}}
	
	\item \textcolor{nie}{Za pomocą RS-232 możemy połączyć ze sobą 2 stacje DCE} \\
	{\small \emph{Połączyć możemy dwie stacje DTE, lub DTE z DCE. Dwie stacje DCE łączą się za pomocą łącza telefonicznego.}}
	
	\item \textcolor{tak}{W MODBUS kontrola błędów jest realizowana za pomocą LRC lub CRC.} \\
	{\small \emph{Tak, LRC wykorzystywane jest w trybie ASCII, CRC w trybie RTU.}}
	
	\item \textcolor{nie}{Do portu RS 485 można podłączyć tylko jedno urządzenie, ale za to obsługiwać go z dużo większą szybkością i na większą odległość niż jest to możliwe w przypadku interfejsu RS 232.} \\
	{\small \emph{Można podłączyć do 32 stacji.}}
	
	\item \textcolor{nie}{Format ramki w protokole Modbus jest następujący: znacznik początku ramki, adres urządzenia slave, adres mastera, pole danych, znacznik końca ramki.} \\
	{\small \emph{Opis nie pasuje ani do trybu ASCII, ani RTU}}
	
	\item \textcolor{nie}{RS 232 jest portem przeznaczonym dla asynchronicznej transmisji znakowej, realizowanej zazwyczaj w trybie dupleksowym, czyli dwukierunkowej transmisji niejednoczenej (naprzemiennej)} \\
	{\small \emph{Tryb dupleksowy jest równoczesny, to półdupleksowy jest niejednoczesny.}}
	
	\item \textcolor{tak}{W interfejsie RS 232 linie TxD i RxD służą do transmisji znaków, natomiast DTR, RTS to wyjścia kontrolne, a DSR, CTS, RI i DCD to wejścia kontrolne.} \\
	{\small \emph{Indeed}}
	
	\item \textcolor{tak}{Multipleksowanie urządzeń ze znakowym portem asynchronicznym pozwala na ich kontrolę poprzez jeden port RS-232.} \\
	{\small \emph{Żeby kontrolować kilka urządzeń z jednego portu potrzebny jest koncentrator. Jeśli "używanie koncentratora" równa się "multipleksowanie", to PRAWDA.}}
	
	\item \textcolor{tak}{Węzeł podrzędny w systemie MODBUS po wykryciu błędu w komunikacie wysyła potwierdzenie negatywne	do węzła nadrzędnego.} \\
	{\small \emph{W odpowiedzi pole to jest wykorzystywane do pozytywnego lub negatywnego potwierdzenia wykonania polecenia.}}
	
	\item \textcolor{tak}{Czy w trybie ASCII systemu MODBUS każdy bajt wysyłany jest jako znak z przedziału 0x00, 0xFF?} \\
	{\small \emph{Bajt dzielimy na 2 części i wysyłamy jako 2 znaki z przedziału 0-9 i Ah-Fh}}
		
	
\end{itemize}


\section{\textcolor{blue}{USB}}
\subsection*{Prawda/Fałsz}
\begin{itemize}
	
	\item \textcolor{tak}{Kontrola urządzenia USB odbywa się poprzez zapisy komunikatów do bufora o numerze 0 i odczycie informacji statusowych z bufora o numerze 0.} \\
	{\small \emph{Zgadza się.}}
	
	\item \textcolor{nie}{W przypadku błędu transmisji każda transakcja USB jest powtarzana, ponieważ niedopuszczalne jest przekazywanie danych przekłamanych.} \\
	{\small \emph{Transakcje izochroniczne nie są powtarzane w przypadku błędu transmisji.}}
	
	\item \textcolor{tak}{Hub nie dopuszcza ruchu full speed do portów, do których są podłączone urządzenia low speed.} \\
	{\small \emph{Tak, urządzenie lowspeed blokuje możliwość włączenia fullspeed na całym porcie.}}
	
	\item \textcolor{nie}{Reset portu USB polega na rekonfiguracji hosta, po której host zapisuje tablicę deskryptorów do urządzenia podłączonego do tego portu.} \\
	{\small \emph{Reset portu USB polega na rekonfiguracji urządzenia. W następującej procedurze enumeracji między innymi dochodzi do odczytu tablicy deskryptorów z urządzenia przez host.}}
	
	\item \textcolor{nie}{Typowa transakcja USB składa się z pakietów żądania i odpowiedzi, z których każdy potwierdzany jest osobnym potwierdzeniem.} \\
	{\small \emph{Typowa transakcja USB składa sie z pakietów token, data i handshake. Transakcje izochroniczne nie są potwierdzane.}}
	
	\item \textcolor{nie}{W systemie USB urządzenia zgłaszają żądania do hosta, który je kolejkuje i następnie obsługuje w kolejności pojawiania się zgłoszenia.} \\
	{\small \emph{Urządzenia nie zgłaszają żądania, tylko są odpytywane przez hosta. Host nie tworzy jednej kolejki, tylko w miarę możliwości stara się obsługiwać wszystkie urządzenia jednocześnie, równomiernie, zapobiegając zawłaszczeniu.}}
	
	\item \textcolor{nie}{W USB można połączyd kaskadowo do 5 hubów, korzystających z zasilania magistralowego} \\
	{\small \emph{Podłączyć je można tylko korzystając z zasilania zewnętrznego lub hybrydowego. Przy zasilaniu magistralowym zabraknie zasilania już na drugim hubie. Co więcej, należy mieć na uwadze maksymalne dopuszczalne opóźnienie sygnału, które przy przejściu przez 5 hubów jest osiągane - 350ns. Urządzenia podpięte do 5'tego huba mogą nie działać poprawnie.}}
	
	\item \textcolor{nie}{Mechanizm data toggle w USB służy do przywracania synchronizacji pomiędzy hostem i urządzeniem, utraconej na skutek wystąpienia błędów w pakietach danych.} \\
	{\small \emph{Mechanizm data toggle zabezpiecza przed utratą synchronizacji pomiędzy hostem i urządzeniem na skutek błędu w potwierdzeniu odsyłanym przez odbiorcę.}}
	
	\item \textcolor{tak}{Host kontroler USB komunikuje się z interfejsem magistrali USB urządzenia peryferyjnego za pomocą fizycznego kanału komunikacyjnego.} \\
	{\small \emph{Tak, używamy kabelka.}}
	
	\item \textcolor{nie}{Kamera internetowa może przesyłać obraz do komputera za pomocą transferu izochronicznego z szybkością LowSpeed w interfejsie USB.} \\
	{\small \emph{Z tabelki można wyczytać, że dla transferu izochronicznego nie można wykorzystać szybkości LowSpeed.}}
	
	\item \textcolor{tak}{Pakiety USB przesyłane z szybkością LowSpeed muszą byd poprzedzone pakietem preambuły} \\
	{\small \emph{Tak, jest on charakterystyczny dla pakietów przesyłanych z szybkością LowSpeed}}
	
	\item \textcolor{nie}{Urządzenie peryferyjne USB 2.0 może być podłączone do host kontrolera za pośrednictwem maksymalnie sześciu hubów.} \\
	{\small \emph{Aby spełnić normę (ograniczenie czasowe oczekiwania na odpowiedź), można podłączyd za pośrednictwem maksymalnie 5 hubów.}}
	
	\item \textcolor{nie}{Pole PID w pakiecie USB zabezpieczone jest 16-bitową sumą kontrolną CRC.} \\
	{\small \emph{Pole PID zabezpieczone jest 4-bitowym polem kontroli, będącym prostą negacją bitów pola PID.}}
	
	\item \textcolor{nie}{Do portu dolnego huba podłączane mogą byd tylko wtyki USB typu B.} \\
	{\small \emph{Tylko wtyki typu A.}}
	
	\item \textcolor{nie}{Transakcja dzielona w USB 1.1 składa sie z dwóch części: SSPLIT i CSPLIT.} \\
	{\small \emph{Takie czary dopiero w USB 2.0}}
	
	\item \textcolor{nie}{W przypadku połączenia USB HighSpeed wykonywane jest podparcie linii D- do Vcc za pośrednictwem rezystora 1,5k.} \\
	{\small \emph{Po podłączeniu urządzenia High Speed wpierw jest ono identyfikowane jako Full Speed, więc wykonywane jest podparcie linii D+ do Vcc za pośrednictwem rezystora 1,5k. Następnie, poprzez chirp ("dwierkanie") host i urządzenie ustalają, czy możliwa jest komunikacja w trybie High Speed. Jeśli tak, usuwane jest podparcie przez rezystor, a obwód zamykany jest terminatorami.}}
	
	\item \textcolor{nie}{W kodowaniu NRZI co sześć jedynek jest wstawiany bit synchronizacji "0".} \\
	{\small \emph{Pomieszane pojęcia. W kodowaniu NRZI nie występuje dodawanie bitu synchronizacji. Proces ten nazywa się bit stuffing. Zdanie było by poprawne, gdyby brzmiało np. W kodowaniu NRZI z bit stuffingiem co sześć.}}
	
	\item \textcolor{nie}{Transakcje kontrolna i przerwaniowa w USB 1.1 są transakcjami aperiodycznymi z gwarantowanym pasmem w ramach jednej mikroramki.} \\
	{\small \emph{Transakcja kontrola jest transakcją aperiodyczną. Transakcja przerwaniowa jest transakcją periodyczną.}}
	
	\item \textcolor{tak}{W kontrolerze OHC transakcje izochroniczne są porządkowane/kolejkowane w drzewo/strukturę drzewiastą.} \\
	{\small \emph{Tak, OHC wykorzystuje strukturę drzewa, a UHC tablicę wskaźników (listę podwieszaną).}}
	
	\item \textcolor{tak}{Standard USB 2.0 wymaga skręconych, ekranowanych kabli.} \\
	{\small \emph{Well, High speed all the way, więc wymaga}}
	
	\item \textcolor{nie}{Transfer kontrolny i przerwaniowy są transferami aperiodycznymi.} \\
	{\small \emph{Było podobne pytanie. Transfer kontrolny jest aperiodyczny, transfer przerwaniowy jest periodyczny.}}
	
	\item \textcolor{tak}{Wielowarstwowa architektura USB 2.0 składa się z 3 warstw.} \\
	{\small \emph{Tak - warstwa interfejsu magistrali USB, warstwa urządzenia USB, warstwa funkcji urządzenia}}
	
	\item \textcolor{tak}{W porcie USB dane są dzielone na transakcje.} \\
	{\small \emph{Dane w ramce są dzielone na transakcje, więc tak}}
	
	\item \textcolor{nie}{Hub podłączony do portu USB ma obciążalność 100uA.} \\
	{\small \emph{Hub podłączony do portu USB bez własnego zasilania (zasilanie magistralowe) ma obciążalnośd dla portów dolnych do 100mA na port (maksymalną 400mA na cały hub). Hub z zasilaniem zewnętrznym lub hybrydowym ma obciążalnośd do 500mA na port.}}
	
	\item {W systemie USB do mechanizmów kontroli danych należą:}
	\begin{itemize}
		\item \textcolor{tak}{Przełączanie pakietów danych} \\
		{\small \emph{Tzw. Data Toggle}}
		
		\item \textcolor{tak}{Wykrywanie braku aktywności na linii danych;}
		
		\item \textcolor{nie}{Zabezpieczenie znacznika SOF lub EOF} \\
		{\small \emph{Reakcją jest natomiast objęte wystąpienie fałszywego znacznika kooca pakietu (false EOP)}}
		
		\item \textcolor{nie}{kodowanie LRC} \\
		{\small \emph{Pakiety zabezpieczone są kodowaniem CRC.}}
	\end{itemize}
	
	\item \textcolor{nie}{Wydajnośd dolnego portu (USB 2.0) wynosi 500mA.} \\
	{\small \emph{Nie wiadomo. Zasilany Hub może wystawić te 500mA, ale niezasilany już tylko 100mA}}
	
	\item \textcolor{tak}{USB 2.0 ma parę przewodów ekranowanych.} \\
	{\small \emph{Taki upgrade.}}
	
	\item \textcolor{nie}{W kodowaniu NZR wstawia się dodatkowe bity synchroniczne.} \\
	{\small \emph{Dodatkowe bity synchroniczne wstawia się w kodowaniu NRZI}}
		
	\item \textcolor{nie}{Urządzenie USB 2.0 może zasygnalizować swoją niegotowość do zapisu danych z szybkością High-Speed wysyłając pakiet PING-NYET.} \\
	{\small \emph{Wychodzi na to, że niegotowość zgłasza samym NYET? Pyta – PING, odpowiada (niegotowość) NYET. I Tak cały czas, chyba że dostanie ACK. ACK – wykonanie transakcji OUT. NYRT – host kontynuuje wysyłanie zapytań PING}}
	
	\item \textcolor{nie}{W systemie deskryptorów urządzenia USB może wystąpić kilka deskryptorów urządzenia, konfiguracji, interfejsów I punktów końcowych.} \\
	{\small \emph{Deskryptor urządzenia może być jeden. Innych – konfiguracji, interfejsu, końcowych może być więcej.}}
	
	\item \textcolor{tak}{Hub USB ma przerwaniowy punkt końcowy, który wykorzystuje do powiadamiania hosta o podłączeniu	urządzenia USB do któregoś z jego portów dolnych.} \\
	{\small \emph{Chyba.}}
	
	\item \textcolor{nie}{Na wierzchołku wielopoziomowego, hierarchicznego układu deskryptorów USB znajduje się deskryptor konfiguracji.}
	{\small \emph{Na szczycie znajduje się pojedynczy deskryptor urządzenia.}}
	
	\item \textcolor{nie}{Transfer masowy I izochroniczny USB 1.1 są przykładami transferów aperiodycznych z zagwarantowanym pasmem w ramach jednej mikroramki.} \\
	{\small \emph{Izochroniczny jest periodyczny, masowy nie ma zagwarantowanego pasma (wg tabelki z	prędkościami)}}
	
	\item \textcolor{nie}{W deskryptorze konfiguracji USB jest jakiś pole statusowe, które mówi o maksymalnym poborze prądu. Dla wartości 50 urządzenie pobiera 50mA.} \\
	{\small \emph{Pole to jest tak skonstruowane, żeby wartość zmieściła się w jednym bajcie, ze skokiem co 2mA. Dlatego urządzenie, które zgłasza, że 50 może zasysać maksymalnie 100mA.}}
	
	\item \textcolor{nie}{Uszeregowanie transakcji w USB nie zależy od implementacji kontrolera.} \\
	{\small \emph{W OHC przerwaniowe są w strukturze drzewa, a w UCH listy podwieszanej, co ma wpływ na uszeregowanie (do sprawdzenia).}}
	
	\item \textcolor{nie}{Host może zasygnalizować chęć zapisu danych do urządzenia wysyłając pakiet NYET do urządzenia USB 2.0, które z kolei odpowiada pakietem PING jeśli jest gotowe do zapisu.} \\
	{\small \emph{To host posyła PING - zapytanie, czy urządzenie jest gotowe do zapisu. Te odsyła ACK - gotowe, lub NYET - jeszcze nie.}}
	
\end{itemize}

\section{\textcolor{blue}{IEEE 1394 Firewire}}
\subsection*{Prawda/Fałsz}
\begin{itemize}
	
	\item \textcolor{tak}{Po resecie w systemie FireWire (IEEE1394) wykonywane są procedury TREEID  i SELFID .} \\
	{\small \emph{Po resecie następuje „TREEID” odpowiedzialne za ustalenie węzła głównego a później „SELFID” odpowiedzialne za rozesłanie adresów do poszczególnych portów.}}
	
	\item \textcolor{tak}{Transakcja dzielona IEEE1394 umożliwia wykorzystanie magistrali przez inne węzły po zakooczeniu subakcji żądania (request), a przed wysłaniem odpowiedzi (response).} \\
	{\small \emph{Tak, pomiędzy żądaniem a odpowiedzą magistrala jest wolna i można ją wykorzystać}}
	
	\item \textcolor{nie}{Transakcje asynchroniczne IEEE1394 są uprzywilejowane w stosunku do transakcji izochronicznych i w związku z tym zawsze wykonywane są na początku cyklu.} \\
	{\small \emph{Jest odwrotnie. To izochroniczne są uprzywilejowane nad asynchronicznymi.}}
	
	\item \textcolor{tak}{Urządzenie IEEE 1394 będące konsumentem zasilania może posiadad co najwyżej jedno 6-kontaktowe gniazdo IEEE 1394.} \\
	{\small \emph{Tak po prostu jest.}}
	
	\item \textcolor{tak}{Numery węzłów IEEE 1394 nadawane są podczas procedury samoidentyfikacji na podstawie wartości wewnętrznych liczników odebranych pakietów SelfID.} \\
	{\small \emph{Tak, SELFID przypisuje każdemu węzłowi unikatowy identyfikator pełniący rolę adresu, a następnie rozsyła je w formie pakietów selfid z każdego portu do wszystkich pozostałych węzłów.}}
	
	\item \textcolor{tak}{Pakiet potwierdzenia odbioru asynchronicznego pakietu żądania zapisu bloku danych w pierwszej fazie transakcji asynchronicznej IEEE 1394 zawiera sumę kontrolną w formie parzystości.} \\
	{\small \emph{Tak, zawiera kod potwierdzenia i parzystość - patrz budowa pakietów}}
	
	\item \textcolor{nie}{W IEEE 1394 pakiet nowego cyklu (SCP) jest wysyłany ZAWSZE co 125us.} \\
	{\small \emph{Izochroniczne owszem, ale asynchroniczne w ramach interwału równych szans, którego długość zależy od liczby węzłów asynchronicznych jednocześnie żądających dostępu do łącza.}}
	
	\item \textcolor{tak}{IEEE 1394 posiada osobne pary ekranowanych przewodów (dla) TPA i TPB.} \\
	{\small \emph{Tak wynika z przekroju budowy}}
	
	\item \textcolor{tak}{Pole adresowe w IEEE1394 składa się z numeru magistrali (10b), numeru węzła (6b) i adresu w węźle (48b).} \\
	{\small \emph{Wszystko się zgadza.}}
	
	\item \textcolor{nie}{W systemie IEEE1394 węzeł A o szybkości S100 połączono z węzłem B o szybkości S200 za pośrednictwem węzła C o szybkości S400, co zwiększyło szybkość transmisji pomiędzy węzłami A i B w stosunku do ich połączenia bezpośredniego.} \\
	{\small \emph{Nie, węzeł A wciąż nadaje z szybkością S100, a teraz dodatkowo musi przejść przez węzeł C}}
	
	\item \textcolor{nie}{W interfejsie IEEE1394 przerwa pomiędzy subakcjami transakcji asynchronicznych jest mniejsza od przerwy pomiędzy transakcjami izochronicznymi, co powoduje ich uprzywilejowanie podczas arbitrażu.} \\
	{\small \emph{Izochroniczne mają pierwszeństwo, to raz. Dodatkowo przerwa pomiędzy izochronicznymi zazwyczaj jest krótsza od tej pomiędzy asynchronicznymi.}}
	
	\item \textcolor{nie}{Biorący udział w arbitrażu asynchronicznym węzeł A (dotyczy interfejsu IEEE 1394) nie może uzyskać dostępu do łącza, bo zdominował je asynchroniczny węzeł B, który ciągle wygrywa arbitraż.} \\
	{\small \emph{Kolejność przydziału zależy od położenia węzła w drzewie systemu. Węzły położone bliżej korzenia uzyskają dostęp przed węzłami bardziej oddalonymi. Łączny czas wykonania subakcji przez wszystkie węzły nazywa się fairness interval. W tym czasie każdy węzeł uzyska dostęp do magistrali, aby wykonać subakcję. Dostęp do magistrali węzeł zarządzający przekazuje “rotacyjnie”. Następna subakcja będzie mogła być wykonana w kolejnym interwale równych szans.}} \\
	
	\item \textcolor{nie}{Odbiorca transakcji asynchronicznej wygrał arbitraż I odesłał pakiet odpowiedzi. Inicjator transakcji odebrał odpowiedź I oczekuje na wygranie arbitrażu w celu odesłania pakietu potwierdzenia.} \\
	{\small \emph{Nie, istnieje coś takiego jak dostęp natychmiastowy - potwierdzenie wysyła się za pośrednictwem warstwy PHY, która nie rywalizuje o dostęp do magistrali.}}
	
	\item \textcolor{tak}{W przypadku transakcji dołączanych (interfejs IEEE 1394) nie trzeba rywalizować o dostęp do magistrali w celu odesłania odpowiedzi.} \\
	{\small \emph{Transakcja dołączana dostarcza dane przy potwierdzeniu, a ono nie wymaga arbitrażu.}}
	
	\item \textcolor{nie}{W interfejsie IEEE 1394 szansa wygrania arbitrażu wzrasta wraz ze wzrostem odległości (mierzone liczbą węzłów pośredniczących) węzła ubiegającego się o dostęp do magistrali od korzenia.} \\
	{\small \emph{Jest na odwrót – bliżej korzenia, większe szanse.}}
	
	\item \textcolor{nie}{Wartość przerw i ograniczeń czasowych w interfejsie IEE 1394 są “na sztywno” określone przez standard i nie mogą być korygowane.} \\
	{\small \emph{Przerwa pomiędzy subakcjami transakcji asynchronicznej może być regulowana.}}
	
	\item \textcolor{nie}{W przykładowym interfejsie IEEE 1394 występują tylko transakcje asynchroniczne. Jeżeli uczestnikiem (inicjatorem lub odbiorcą) transakcji asynchronicznej jest korzeń, to zawsze wygra on arbitraż jako pierwszy.} \\
	{\small \emph{Zgodnie z zasadą "Bliżej korzenia, większe szanse" korzeń wygrywa wszystko. Jednak występują transakcje izochroniczne i asynchroniczne.}}
	
	\item \textcolor{tak}{Kontroler cyklu musi być korzeniem w topologii IEE1394, bo musi wysyłać sygnał CSP.} \\
	{\small \emph{Chyba.}}
	
	\item \textcolor{tak}{Węzeł w IEE1394, który zainicjalizował transakcje asynchroniczną może być odbiorcą transakcji zainicjalizowanej przez inny węzeł w tym samym Interwale Równych Szans.} \\
	{\small \emph{Raczej tak, węzły asynchroniczne do wykonania transakcji nie wymagają alokacji pasma (subakcja). Also - zarządzanie magistralą stara się umożliwić jednoczesne jej wykorzystanie przez różne transfery.}}
	
\end{itemize}


\section{\textcolor{blue}{IEEE-488 i SCPI}}
\subsection*{Prawda/Fałsz}
\begin{itemize}
	
	\item \textcolor{tak}{GPIB (IEEE-488) jest interfejsem równoległym, opartym na 8-bitowej, 2 kierunkowej magistrali danych i 8 sygnałach sterujących: REN, IFC, ATN, SRQ, EOI, NRFD, NDAC, DAV} \\
	{\small \emph{Tak, bity wysyła się ósemkami, stąd m. in. podaje się prędkość w bajtach na sekundę}}
	
	\item \textcolor{nie}{SCPI to język programowania na bazie języka C wyposażony w biblioteki funkcji sterujących urządzeniami pomiarowo-kontrolnymi.} \\
	{\small \emph{SCPI jest językiem kontroli urządzeo (i nie bazuje na C).}}
	
	\item \textcolor{tak}{Znak ':' w rozkazach SCPI reprezentuje przejście pomiędzy poziomami w rozgałęzionej strukturze subsystemu, natomiast prefiks '*' oznacza rozkaz wspólny.} \\
	{\small \emph{Tak, dwukropek służy do precyzowania zapytania, gwiazdka jako nagłówek komunikatu wspólnego.}}
	
	\item \textcolor{nie}{System statusowy urządzenia SCPI składa się tylko z jednego, 8-bitowego rejestru, w którym bit B6 jest zgłoszeniem żądania obsługi.} \\
	{\small \emph{Składa się z minimum dwóch rejestrów, których układ jest wielopoziomowy (hierarchiczny).}}
	
	\item \textcolor{nie}{Kontrola szeregowa I kontrola równoległa to mechanizmy automatycznego wykrywania urządzeń podłączonych do systemu IEEE 488.} \\
	{\small \emph{Kontrola szeregowa I równoległa służą do identyfikacji urządzeń zgłaszających żadanie obsługi.}}
	
	\item \textcolor{nie}{Maska związana z bajtem statusowym SCPI służy do blokowania ustawiania wybranych bitów bajtu statusowego.} \\
	{\small \emph{maską związaną z bitem statusowym jest rejestr maski żądania obsługi, który odpowiada za selekcję bitów powodujących zgłoszenie żądania, ale nie ma on wpływu na stan samych bitów w bajcie statusowym. Bajt statusowy jest rejestrem zbiorczym swoich rejestrów nadrzędnych, więc jego wartość zależy do wartości tamtych rejestrów i ich masek.}}
	
\end{itemize}

\section{\textcolor{blue}{Inne}}
\subsection*{Prawda/Fałsz}
\begin{itemize}
	\item \textcolor{nie}{Interfejsy USB, IEEE1394 oraz RS-232 udostępniają zasilanie systemowe i mają mechanizmy zarządzania zasilaniem.} \\
	{\small \emph{USB i IEEE-1394 owszem, ale nie RS-232.}}
	
\end{itemize}

\pagebreak
\part{Opracowanie materiałów}

\section{RS-232 – szeregowy port znakowy}
	\subsection{Co to jest?}
	Standard RS-232 został wprowadzony w 1962 r. w celu normalizacji interfejsu pomiędzy \textit{urządzeniem końcowym dla danych} (DTE - Data Terminal Equipment), a \textit{urządzeniem komunikacyjnym} (DCE - Data Communication Equipment). Na zajęciach zajmujemy się tak naprawdę zrewindowaną wersją standardu: RS-232C, wprowadzoną w 1969 roku.\\
	RS-232C umożliwia przesył danych na niewielkie odległości - do 15 metrów - oraz niewielką szybkość - do 20 kb/s - przez niesymetryczne łącze.
	\subsection{Charakterystyka interfejsu RS-232}
		Łącze szeregowe przeznaczone do asynchronicznej transmisji znakowej realizowanej zazwyczaj w trybie półdupleksowym.
		\subsubsection{Transmisja danych}
		Szeregowa asynchroniczna transmisja znakowa w trybie półdupleksowym (praktycznie tylko w tym trybie, ale mogą być inne). CIEKAWOSTKA: ma budowę dupleksową.
		\subsubsection{Rodzaje transmisji}
		\begin{itemize}
			\item Szeregowa - sekwencyjne przesyłanie bitów w ustalonej kolejności (od LSD lub MSB) po jednej linii transmisyjnej.\\
			\includegraphics[width=9cm]{./wyklady/RS232_2_1.pdf}
			\item Równoległa - przesyłanie bitów słowa po przyporządkowanej każdemu bitowi linii transmisyjnej (bity przesyłane równolegle, słowa przesyłane szeregowo).\\
			\includegraphics[width=9cm]{./wyklady/RS232_2_2.pdf}
		\end{itemize}
		\subsubsection{Definicje danych}
		\begin{itemize}
			\item "0" - 0V
			\item "1" - 12V
			\item Czas transmisji jednego bitu T - stały, nie większy niż czas propagacji.
			\item $\frac{1}{T}$ - liczba bitów przesyłana w jednostce czasu. Standardowe wartości 110, 150, 300, 600, 1200, 2400 ... [b/s]
			\item Impulsy rozeznające - sprawdzają stan bitów odebranych (następują co T, które narzuca nadawca).
		\end{itemize}
		\subsubsection{Jednostka informacyjna - znak}
		Jednostka informacji o ściśle określonym formacie. Odbiorca dysponuje impulsami próbkującymi, które rozpoznają stan sygnału (odpytują).\\
		\includegraphics[width=10cm]{./wyklady/RS232_3_2.pdf}\\
		\subsubsection{Przekazywanie konfiguracji}
			Aby nadawca i odbiorca mogli się porozumieć i interpretować znaki w ten sam sposób, muszą zostać tak samo skonfigurowane. Innymi słowy, muszą posiadać ten sam takt nadawania.\\Metody:\\
			\begin{itemize}
				\item dodatkowe łącze
				\item jako element konfiguracji (wykorzystanie generatorów kwarcowych, synchronizm częstotliwościowy)
			\end{itemize}
			Nominalne położenie impulsu = ok. $\frac{1}{2}\times{T}$ - pośrodku, największe bezpieczeństwo próbkowania. Służą do tego układy korekcji fazy impulsu - liczniki, zliczają liczbę impulsów na wejściu i dają 1 na wyjściu.
		\subsubsection{Definicja znaku}
		\begin{itemize}
			\item \textbf{START} - bit kontrolny, znacznik początku (SOF - Start Of Frame) - jałowy z punktu widzenia przesyłanej informacji i służący jedynie w celu synchronizacji. START zapewnia $\frac{n}{2}$ jako stan licznika (fazy impulsu).
			\item \textbf{DANE} - 7-8 bitów (kiedyś też 5-6, obecnie już nieużywane), które są treścią znaku, począwszy od bitu najmniej znaczącego (LSB - least significant bit). Tym bitem jest B0.
			\item \textbf{PARITY} – bit kontroli poprawności znaku, służy jako zabezpieczenie informacji. Może, ale nie musi występować. Jednak decyzja o jego występowaniu ma charakter globalny - dotyczy każdego znaku w danej transmisji. Jego stan określa zasada:
			\begin{itemize}
				\item Kontrola parzystości (Even parity) - polega na sprawdzeniu liczby jedynek na polu danych i ustawieniu bitu kontrolnego na "1" w przypadku nieparzystej liczby jedynek lub na "0" w przypadku parzystej (uzupełnienie do parzystości).
				\item Kontrola nieparzystości (Odd parity) - polega na sprawdzeniu liczby zer na polu danych i ustawieniu bitu kontrolnego na "1" w przypadku nieparzystej liczby zer lub na "0" w przeciwnym przypadku.
				\item Brak kontroli (None)
			\end{itemize}
			Ten bit kontroli pozwala wykryć przekłamanie w transmisji danych pod warunkiem, ze liczba przekłamań jest nieparzysta.
			\item \textbf{STOP} – 1 lub 2 bity kontrolne, znacznik końca znaku.
		\end{itemize}
		\subsubsection{Konwencja nazewnicza rodzajów transmisji}
			[Ilość bitów danych][Rodzaj kontroli][Liczba bitów stopu]\\Przykłady:
			\begin{itemize}
				\item 7E2 - 7 bitów danych, kontrola parzystości, 2 bity stopu (10 bitów + START = 11 bitów)
				\item 8O1 - 8 bitów danych, kontrola nieparzystości, 1 bit stopu (11 bitów)
				\item 8N2 - 8 bitów danych, brak kontroli, 2 bity stopu (11 bitów)
			\end{itemize}
		\subsubsection{Rodzaje transmisji}
		\begin{itemize}
			\item \textbf{Synchroniczna} - elementy informacji wysyłane w takt zegara nadajnika. W ten sposób przesyłane są bity w ramach pojedynczej jednostki informacyjnej.
			\item \textbf{Asynchroniczna} - wysyłanie elementów informacji niesynchronizowane zegarem nadajnika. W ten sposób są wysyłane poszczególne jednostki - ich wprowadzanie nie jest sygnalizowany żadnym sygnałem, więc odstęp między nimi jest dowolny.\\
			Czas trwania bitu nazywa się \emph{odstępem jednostkowym} i oznaczamy go $t_{b}$. Jego odwrotność ($f=\frac{1}{t_{b}}$) określa szybkość transmisji w bodach, gdzie 1 [bd] = 1 [bit/s].
		\end{itemize}
		\subsubsection{Transmisja w RS-232}
		\includegraphics[width=9cm]{./wyklady/RS232_4_1.pdf}
		\begin{itemize}
			\item Synchroniczne wysyłanie bitów
			\item Asynchroniczne wysyłanie znaków
			\begin{itemize}
				\item Polega na wysyłaniu pojedynczych znaków, które mają ścisłe określony format.
				\item Brak sygnału zegarowego określającego momenty wysyłania znaków.
				\item Odstępy między znakami nieokreślone.
			\end{itemize}
		\end{itemize}
		\subsubsection{Tryby transmisji}
		\begin{itemize}
			\item \textbf{Simpleksowa} - jednokierunkowa, z nadajnika do odbiornika.
			\item \textbf{Półdupleksowa} (HDX) - dwukierunkowa, niejednoczesna (w danej chwili czasu jedno urządzenia jest nadajnikiem, a drugie odbiornikiem). Zakłada istnienie tylko jednej linii transmisyjnej. Wymaga konfiguracji (informacja, kto kiedy nadaje).
			\item \textbf{Dupleksowa} (FDX) - dwukierunkowa, jednoczesna (w danej chwili czasu oba urządzenia mogą spełniać rolę nadajnika lub odbiornika). Brak konieczności sprawdzania czy łącze jest wolne oraz mechanizmu rezerwacji łącza.
		\end{itemize}
	\subsection{Komunikacja DTE-DCE - sygnały w porcie RS-232}
	Komunikacja dwóch stacji DTE przez komutowane łącze telefoniczne.\\
	\includegraphics[width=12cm]{./wyklady/RS232_6_1.pdf}\\
	\begin{table}[h]
		\begin{tabular}{|c|c|c|c|c|}
			\hline
			\multicolumn{5}{|c|}{\textbf{Urządzenia}}                        \\ \hline
			\multicolumn{1}{|c|}{DTE} & \multicolumn{2}{|c|}{Data Terminal Equipment}      & \multicolumn{2}{|c|}{Komputer}           \\ \hline
			\multicolumn{1}{|c|}{DCE} & \multicolumn{2}{|c|}{Data Communication Equipment} & \multicolumn{2}{|c|}{Modem}              \\ \hline
			\multicolumn{5}{|c|}{\textbf{Linie} (sygnały)}                   \\ \hline
			\textbf{Skrót} & \textbf{Nazwa}	   & \textbf{Znaczenie} & \textbf{Przeznaczenie} & \textbf{Kierunek} \\ \hline
			TxD & Transmitted Data             & Dane nadawane      & Linia danych		& Wyjście	\\ \hline
			RxD & Received Data                & Dane odbierane     & Linia danych		& Wejście	\\ \hline
			DTR & Data Terminal Ready          & Gotowość DTE       & Linia kontrolna	& Wyjście	\\ \hline
			DSR & Data Set Ready               & Gotowość DCE       & Linia kontrolna	& Wejście	\\ \hline
			RTS & Request to Send              & Żądanie nadawania  & Linia kontrolna	& Wyjście	\\ \hline
			CTS & Clear To Send                & Zgoda na nadawanie & Linia kontrolna	& Wejście	\\ \hline
			RI  & Ring Indicator               & Wskaźnik wywołania & Linia kontrolna   & Wejście	\\ \hline
			DCD & Data Carrier Detected        & Wykrycie nośnej    & Linia kontrolna   & Wejście	\\ \hline
			SG  & Signal Ground                & Masa sygnałowa     & Masa				& ------	\\ \hline
		\end{tabular}
	\end{table}
		\subsubsection{Fazy pracy układu}
		\begin{itemize}
			\item Tryb nawiązywania połączenia
			\item Tryb transmisji danych (wtedy nas interesują dupleksy i inne)
		\end{itemize}
		\subsubsection{Linie w złączu RS-232}
		\begin{itemize}
			\item Linie danych: TxD, RxD
			\item Linie kontrolne: DTR, DSR, RTS, CTS, RI, DCD
		\end{itemize}
	\subsection{Połączenie bezmodemowe DTE-DTE}
	Przykład połączenia dla transmisji dupleksowej.\\
	\includegraphics[width=9cm]{./wyklady/RS232_7_1.pdf}
	\begin{itemize}
		\item PG, SG - masa
		\item TxD, RxD - dane
		\item RTS, CTS, DCD, DSR, DTR - sterowanie
	\end{itemize}
	\subsection{Kontrola transmisji: handshake i protokół XON/XOFF}
		\subsubsection{Handshake}
		\includegraphics[width=9cm]{./wyklady/RS232_9_1.pdf}
		\begin{itemize}
			\item DTR = 1 - zgoda na nadawanie
			\item DTR = 0 - brak zgody na nadawanie
			\item DTR informuje, czy bufor jest zapełniony. DSR sprawdza go u partnera przed wysłaniem dalszych danych.
			\item Analogiczna sytuacja, kiedy podłączone są RTS i CTS zamiast DTR i DSR. RTS wystawia informację, CTS sprawdza.
		\end{itemize}
		\subsubsection{Protokół XON/XOFF}
		Występuje przy wymianie informacji w trybie dupleksowym. Umożliwia blokowanie i odblokowywanie transmisji danych. Np. drukarka - gdy skończy się papier w trakcie drukowania, przesył jest blokowany, Gdy użytkownik uzupełni papier, transmisja jest wznawiana. Taki protokół XON/OFF nazywany jest programowym (software XON/OFF). {\small Rozwiązanie hardware to transmisja półdupleksowa za pośrednictwem sygnałów w kanale wtórnym.}\\
		\includegraphics[width=9cm]{./wyklady/RS232_9_2.pdf}
		\begin{itemize}
			\item XON – ASCII 19 (CTRL-S)
			\item XOFF – ASCII 17 (CTRL-Q)
		\end{itemize}
	\subsection{Parametry elektryczne sygnałów}
	Poniżej przestawiono schemat "obwodu stykowego" złożonego ze źródła sygnału, toru transmisyjnego i odbiornika. Parametry zdefiniowano przy założeniu, że szybkość transmisji nie przekracza 20 kbd.
	\includegraphics[width=9cm]{./wyklady/RS232_10_1.pdf}\\
	Na rysunku powyżej: kiloomy oraz mikrosekundy.\\
	Wada: jest to obwód represyjny, da się go silnie zakłócić poprzez różnicę potencjałów pomiędzy masami.
	\subsection{Standardy RS-423, RS-422, RS-485}
	Niesymetryczna przesyłanie danych w RS-232C ogranicza szybkość i odległość transmisji, a ponadto nie jest zabezpieczone przed zakłóceniami zewnętrznymi. Aby to polepszyć wymyślono inne standardy.
		\subsubsection{RS-423A}
		\includegraphics[width=9cm]{./wyklady/RS232_12_1.pdf}
		\begin{itemize}
			\item szybkość do 100 kbd (przy zasięgu do 30 m)
			\item zasięg do 1200 m (przy szybkosci do 3 kbd)
		\end{itemize}
		Standard RS-423A określa elektryczną charakterystykę napięciowego obwodu transmisyjnego złożonego z niesymetrycznego nadajnika oraz symetrycznego (różnicowego) odbiornika. Takie obwody stosuje się do przesyłania sygnałów binarnych pomiędzy DTE i DCE, które reprezentują dane lub funkcje sterujące.\\
		Zastosowanie różnicowego obciążenia pozwala na znaczne zmniejszenie wpływu napięcia wspólnego $U_{G}$ powstałego na wskutek różnicy potencjałów masy nadajnika i odbiornika, jak również przesłuchów między nimi.\\
		Standard wymaga aby dla każdego kierunku transmisji istniał przynajmniej jeden niezależny przewód powrotny.\\
		Typowa prędkość wynosi 100 kb/s przy odległości do 30 m.
		\subsubsection{RS-422A}
		\includegraphics[width=9cm]{./wyklady/RS232_12_2.pdf}
		\begin{itemize}
			\item szybkość do 10 Mbd (przy zasięgu do 100 m)
			\item zasięg do 1200 m (przy szybkości 100 kbd)
		\end{itemize}
		Wykorzystuje pełną symetryzację łącza, zapewnia szybka transmisję w obecności zakłóceń. Standardy RS-423 oraz RS-485 określają symetryczny, zrównoważony system transmisji danych złożony z:
		\begin{itemize}
			\item różnicowego nadajnika
			\item dwuprzewodowego zrównoważonego toru przesyłowego
			\item odbiornika o różnicowym obwodzie wejściowym.
		\end{itemize}
		Standard RS-422A nie wprowadza ograniczeń na minimalną i maksymalną częstotliwość, a jedynie na zależność między szybkością zmian sygnału, a czasem trwania bitu.
		\subsubsection{RS-485A}
		Standard RS-485A jest rozwinięciem RS-422. Łącze RS-485A jest również zrównoważone i symetryczne, przy czym dopuszcza się nie tylko wiele odbiorników, ale i wiele nadajników podłączonych do jednej linii. Nadajniki muszą być trójstanowe.\\
		\includegraphics[width=10cm]{./wyklady/RS232_13_1.pdf}
	\subsection{Systemy komunikacyjne oparte na łączu znakowym}
		\subsubsection{System oparty na szeregowym łączu znakowym}
		Podłączenie urządzenia RS-232 do portu COM z pom. int. RS-485\\
		\includegraphics[width=9cm]{./wyklady/RS232_14_1.pdf}\\
		R\tiny T\normalsize - Rezystory zabezpieczające przed niekorzystnym odbiciem fali (tzw. Terminatory).\\
		\textbf{Problem}: dostęp do magistrali kontrolera i urządzeń systemu\\
		\textbf{Rozwiązanie}: Implementacja protokołu komunikacyjnego (warstwa łącza danych). Komputer zarządza innymi urządzeniami w całym systemie. Do zbudowania tego wystarczają proste przejściówki do zmian sygnałów.\\
		\textbf{Komunikacja}:
		\begin{itemize}
			\item Selekcja urządzenia kontrolującego (master) - generuje on rozgłoszenie (broadcast) do wszystkich urządzeń i zbiera dane.
			\item Selekcja urządzenia odbierającego - konieczna gdy wiele urządzeń chce przesłać odpowiedź do mastera, co może powodować konflikt.
			\begin{itemize}
				\item nadanie identyfikatorów (adresacja urządzeń)
				\item zastosowanie przejściówek - są inteligentne i odpowiadają za dostęp do urządzenia.
			\end{itemize}
		\end{itemize}
		\subsubsection{System oparty na łączu znakowym}
		\includegraphics[width=10cm]{./wyklady/RS232_15_1.pdf}\\
		Koncentrator zawiera 4 klucze portu RS. Dostęp jest tylko do jednego wyjścia naraz.\\
		\textbf{Kaskadowe połączenie} - przełącznik podłączony do przełącznika. Pojawia się problem wyboru drogi do urządzenia, która musi być znana. Koncentratory muszą mieć informacje o \textbf{mapie urządzeń}.
	\subsection{System MODBUS}
		Interfejs MODBUS został opracowany w firmie Modicon i jest przyjętym standardem w dla asynchronicznej, znakowej wymiany informacji pomiędzy urządzeniami systemów pomiarowo-kontrolnych.
		\subsubsection{Charakterystyka}
			\begin{itemize}
				\item Reguła dostępu do łącza na zasadzie Master-Slave. 
				\item Zabezpieczenie przesyłanych komunikatów przed błędami
				\item Potwierdzenie wykonania rozkazów zdalnych i sygnalizacja błędów
				\item Mechanizmy zabezpieczające przed zawieszeniem systemu
				\item Wykorzystanie asynchronicznej transmisji znakowej zgodnej z RS-232C
			\end{itemize}
		\includegraphics[width=9cm]{./wyklady/RS232_16_1.pdf}\\
		\subsubsection{Transakcja}
		Jedno urządzenie może inicjować transakcje (master), a pozostałe (slave) odpowiadają jedynie na zapytania mastera. Transakcja składa się z polecenia (query) wysyłanego z master do slave oraz z odpowiedzi (response) przesyłanej ze slave do master. Odpowiedź zawiera dane żądane przez master lub potwierdzenie realizacji jego połączenia. Wykrycie końca kończy fazę w której następuje przekazanie łącza masterowi.\\
		\subsubsection{Format wiadomości}
		Dotyczy zarówno poleceń jednostki nadrzędnej, jak i odpowiedzi podrzędnych.
		\begin{itemize}
			\item Adres
			\item Kod funkcji reprezentujący rozkaz, pierwszy bit rozkazu oznacza jego rodzaj. 0 - normalny, 1 - szczególny.
			\item Dane
			\item Kontrola błędów (dla pracy w warunkach przemysłowych)
		\end{itemize}
		W przypadku odpowiedzi odpowiednio w polach znajdują się:
		\begin{itemize}
			\item Adres (swój, slave'a, do kontroli poprawności)
			\item Pole potwierdzenia realizacji rozkazu
			\item Dane żądane przez master
			\item Kontrola błędów
		\end{itemize}
		\subsubsection{Rodzaje transakcji}
			\begin{itemize}
				\item \textbf{Adresowana} - przeznaczona dla pojedynczej jednostki slave
				\item \textbf{Rozgłoszeniowa} (broadcast) - wysyłana do wszystkich jednostek podrzędnych. Na ten rodzaj polecenia jednostki nie przesyłają odpowiedzi.
			\end{itemize}
		\subsubsection{Rodzaje odpowiedzi}
			\begin{itemize}
				\item \textbf{Normalna} - w przypadku poprawnego wykonania polecenia.
				\item \textbf{Szczególna} - jeżeli slave wykryje błąd przy odbiorze wiadomości lub nie jest w stanie wykonać polecenia, to przygotowuje specjalny komunikat o wystąpieniu błędu i przesyła jako odpowiedź. W przypadku tej wiadomości jest ona \textbf{powiększona} o 128 - miejsce na kod błędu.
			\end{itemize}
		\subsubsection{Parametry protokołu}
			\begin{itemize}
				\item Reguła dostępu do łącza: Master-Slave
				\item Zakres adresów: 1 - 247 (identyfikatory slave'ów)
				\item Adres rozgłoszeniowy: 0, rozpoznawany przez wszystkie slave'y
				\item Kontrola błędów: LRC/CRC, ograniczenie czasowe odpowiedzi
				\item Wymagana ciągłość przesyłania znaków w ramce
			\end{itemize}
		\subsubsection{Ramka w systemie MODBUS}
			W systemie MODBUS wiadomości są zorganizowane w ramki o określonym początku i końcu. Umożliwia do odbiornikowi odrzucenie ramek niekompletnych i sygnalizację błędów.
		\subsubsection{Rodzaje transmisji ramek}
			\begin{itemize}
				\item ASCII
				\item RTU
			\end{itemize}
		\subsubsection{Ramka w trybie ASCII}
			Każdy bajt wiadomości przesyłany jest w postaci dwóch znaków ASCII. Zaletą tego rozwiązania jest to, że pozwala na długie odstępy między znakami (1 s) bez powodowania błędów.\\\\\textbf{Format znaku}\\
			\includegraphics[width=10cm]{./wyklady/RS232_18_1.pdf}
			\begin{itemize}
				\item System kodowania: heksadecymalny, znaki ASCII 0-9, A-F. Jeden znak heksadecymalny zawarty jest w każdym znaku ASCII wiadomości.
				\item Jednostka informacyjna: ograniczona znakami start (na początku) i stop (na końcu), 10-bitowa.
				\item Znacznikiem początku ramki jest znak dwukropka (":" - ASCII 3Ah).
				\item Dopuszczalne znaki dla pozostałych pól (poza znacznikiem końca ramki) to 0-9, A-F.
				\item Pole funkcji: dwa znaki w trybie ASCII
				\item Podsumowując: wykorzystujemy 2 znaki heksadecymalne do przesyłu 1go znaku ASCII. Dzielimy ten na dwie części i przesyłamy w dwóch pakietach po 10 bitów.
				\item Urządzenie po wykryciu znacznika początku sprawdza czy pole adresowe zawiera jego własny adres. Jeżeli tak, to odczytuje zawartość pola funkcji i pola danych.
				\item Pole kontrolne LRC (1-bitowe) zabezpiecza część informacyjną. Sumuje cześć informacyjną bajtu i uzupełnia do 2.
				\item Ramka kończy się przesłaniem dwóch znaków: CR i LF.
				\item Ramkę kończy przerwa czasowa trwająca co najmniej $3.5\times$(długości znaku)
				\item Ramki muszą być przesyłane w postaci ciągłej, tzn. odstęp między kolejnymi znakami tworzącymi ramkę nie może być większy niż $1.5\times$(długości znaku).
			\end{itemize}
			Stosowane jest zabezpieczenie części informacyjnej ramki kodem LRC (Longitudinal Redundancy Check).
		\subsubsection{Ramka w trybie RTU}
			\includegraphics[width=10cm]{./wyklady/RS232_18_2.pdf}\\
			W trybie RTU wiadomości zaczynają się odstępem czasowym trwającym minimum $3.5\times$(\emph{czas trwania pojedynczego znaku}), w którym panuje cisza na łączu (można to zrealizować np. przez odmierzanie czasu trwania znaku przy zadanej na łączu szybkości bodowej).
			\begin{itemize}
				\item Pierwszym polem informacyjnym jest adres urządzenia
				\item Dopuszczalne znaki w ramach pól ramki: 0-9, A-F
				\item Zakres kodów operacji: 1 - 255
				\item Urządzenia stale monitorują magistralę. Jak adres odebrany w wiadomości zgadza się z ich własnym, to lecą dalej.
				\item Ramkę kończy przerwa czasowa trwająca co najmniej $3.5\times$(długości znaku)
				\item W przypadku gdy nowa wiadomość pojawia się przed upływem niezbędnej przerwy to będzie ona potraktowana jako kontynuacja poprzedniej wiadomości. Doprowadzi to do błędu sumy kontrolnej.
				\item Ramki muszą być przesyłane w postaci ciągłej, tzn. odstęp między kolejnymi znakami tworzącymi ramkę nie może być większy niż $1.5\times$(długości znaku).
				\item Przekroczenie odstępu powoduje uznanie ramki za niekompletną i błędną.
				\item Kontrola danych typu CRC - 2-bajtowe, silniejsze niż LRC.
			\end{itemize}
		\subsubsection{Warstwa fizyczna}
		\begin{itemize}
			\item Asynchroniczna transmisja znakowa
			\item Formaty znaków
			\begin{itemize}
				\item Tryb ASCII: 7E1, 7O1, 7N2
				\item Tryb RTU: 8E1, 8O1, 8N2
			\end{itemize}
			\item Szybkość: od 1200 bd do 19200 bd
			\item Rodzaj łącza:
			\begin{itemize}
				\item Magistrala RS-485
				\item Multipleksowany RS-232
			\end{itemize}
			\item Rodzaj transmisji (zależny od łącza):
			\begin{itemize}
				\item różnicowa dla RS-485
				\item odniesiona do masy dla RS-232
			\end{itemize}
		\end{itemize}
	\subsection{Kontroler RS-232 w komputerze PC}
	
\section{USB – Uniwersalny interfejs szeregowy}
\section{IEEE-488 and SCPI standards}
\section{IEEE-1394 (FireWire)}
\section{Tłumienie zakłóceń w rozproszonych systemach komputerowych}
\end{document}